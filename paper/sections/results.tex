% Results - BDCC Format
% Transformer-only comparisons following paper_reqs.md structure

%-------------------------------------------------------------------
% 3.1 BASELINE TRANSFORMER RESULTS
%-------------------------------------------------------------------
\subsection{Baseline Transformer Performance}
\label{subsec:baseline}

We first establish baseline performance using single-stream transformer architectures. Table~\ref{tab:baseline} presents results for transformers with and without including our special attention mechanisms, both using Kalman-fused 7-channel input processed through a single Conv1D projection.

\begin{table}[H]
\caption{Baseline transformer performance with Kalman-fused input (7ch). All models use single-stream architecture with embed\_dim=64. 21-fold LOSO-CV on young subjects. }{\color{blue} Need to explain SE and TAP here, people already forgotten what they are when they read to here}
\label{tab:baseline}
\centering
\begin{tabular}{lcccccc}
\toprule
\textbf{Model} & \textbf{SE} & \textbf{TAP} & \textbf{Test F1} & \textbf{Test Acc} & \textbf{Val F1} & \textbf{$\Delta$F1} \\
\midrule
Transformer (no attention) & \ding{55} & \ding{55} & 88.52 & 83.67 & 94.21 & --- \\
Transformer + SE & \ding{51} & \ding{55} & 89.15 & 84.23 & 94.89 & +0.63 \\
Transformer + TAP & \ding{55} & \ding{51} & 88.34 & 83.41 & 93.95 & $-$0.18 \\
\textbf{Transformer + SE + TAP} & \ding{51} & \ding{51} & \textbf{89.80} & \textbf{84.96} & 95.97 & \textbf{+1.28} \\
\bottomrule
\end{tabular}
\end{table}

The combination of SE channel attention and TAP temporal pooling yields +1.28\% F1 improvement over the baseline transformer. SE attention alone provides +0.63\% F1, while TAP alone shows marginal degradation, suggesting that temporal attention benefits from channel-recalibrated features.

%-------------------------------------------------------------------
% 3.2 DUAL-STREAM VS SINGLE-STREAM (KALMAN INPUT)
%-------------------------------------------------------------------
\subsection{Dual-Stream vs. Single-Stream Architecture: Kalman Input}
\label{subsec:stream_kalman}

Having established baseline performance, we now compare single-stream and dual-stream architectures using Kalman-fused orientation input. Table~\ref{tab:stream_kalman} presents this comparison with fixed embedding dimension (64 total, split 32:32 for dual-stream).

\begin{table}[H]
\caption{Single-stream vs. dual-stream with Kalman-fused input (7ch: SMV + acc + orientation). Dual-stream uses separate Conv1D projections with 32:32 embedding split. All models include SE+TAP attention.}
\label{tab:stream_kalman}
\centering
\begin{tabular}{lccccc}
\toprule
\textbf{Architecture} & \textbf{Embed Split} & \textbf{Test F1} & \textbf{Test Acc} & \textbf{Val F1} & \textbf{$\Delta$F1} \\
\midrule
Single-Stream Kalman & 64 (shared) & 89.80 & 84.96 & 95.97 & --- \\
\textbf{Dual-Stream Kalman} & 32:32 & \textbf{91.10} & \textbf{87.30} & 96.10 & \textbf{+1.30} \\
\bottomrule
\end{tabular}
\end{table}

Dual-stream processing provides +1.30\% F1 improvement over single-stream when using Kalman-fused input. This confirms that separate input projections prevent cross-modal interference, allowing the model to extract discriminative features from both acceleration and orientation streams. Figure~\ref{fig:comparison} visualizes the performance comparison across architectures. {\color{blue} Why is the top the same across all experiments?}

\begin{figure}[H]
\centering
\includegraphics[width=0.8\textwidth]{figures/f1_bar_comparison.png}
\caption{\textbf{F1 score comparison across architecture configurations (21-fold LOSO-CV).} The dual-stream Kalman transformer achieves the highest test F1 score of \textbf{91.10\%}, demonstrating the synergistic benefit of combining Kalman fusion with dual-stream processing. Key observations: (1) Adding raw gyroscope to accelerometer \textit{hurts} performance (Single Raw 6ch vs Acc-only), confirming gyroscope noise contamination; (2) Kalman fusion alone improves performance (+0.84\% for single-stream); (3) Dual-stream processing amplifies the Kalman benefit (+3.52\% vs +0.84\%). Error bars show standard deviation across folds, indicating consistent improvement across subjects.}
\label{fig:comparison}
\end{figure}

%-------------------------------------------------------------------
% 3.3 DUAL-STREAM VS SINGLE-STREAM (RAW INPUT)
%-------------------------------------------------------------------
\subsection{Dual-Stream vs. Single-Stream Architecture: Raw Input}
\label{subsec:stream_raw}

To isolate the effect of dual-stream architecture from Kalman fusion, we repeat the comparison using raw gyroscope input instead of Kalman-fused orientation. Table~\ref{tab:stream_raw} presents these results.

\begin{table}[H]
\caption{Single-stream vs. dual-stream with raw input (6ch: acc + raw gyro). Dual-stream uses separate Conv1D projections. All models include SE+TAP attention.}
\label{tab:stream_raw}
\centering
\begin{tabular}{lccccc}
\toprule
\textbf{Architecture} & \textbf{Embed Split} & \textbf{Test F1} & \textbf{Test Acc} & \textbf{Val F1} & \textbf{$\Delta$F1} \\
\midrule
Single-Stream Raw & 64 (shared) & 88.96 & 84.55 & 92.83 & --- \\
Dual-Stream Raw & 32:32 & 87.58 & 83.12 & 92.47 & $-$1.38 \\
\bottomrule
\end{tabular}
\end{table}

Dual-stream architecture with raw gyroscope input shows \textit{decreased} performance ($-$1.38\% F1) compared to single-stream. We attribute this degradation to the following mechanism: raw gyroscope signals are inherently noisy due to MEMS sensor characteristics. Dedicating separate capacity to encode these noisy signals without the denoising benefit of Kalman fusion amplifies noise rather than extracting useful features.

This instability is quantified by cross-fold standard deviation: raw dual-stream exhibits $\sigma = 4.2\%$ compared to $\sigma = 2.8\%$ for Kalman dual-stream, indicating that noisy input leads to less consistent generalization across subjects.

This finding highlights a critical interaction effect: \textbf{dual-stream architecture requires high-quality input streams to provide benefit}. When the orientation stream contains noise (raw gyro), separate processing amplifies rather than mitigates the problem.

%-------------------------------------------------------------------
% 3.4 KALMAN VS RAW INPUT COMPARISON
%-------------------------------------------------------------------
\subsection{Kalman Fusion vs. Raw Gyroscope Input}
\label{subsec:kalman_vs_raw}

Table~\ref{tab:kalman_vs_raw} directly compares Kalman-fused orientation against raw gyroscope input, controlling for architecture.

\begin{table}[H]
\caption{Effect of Kalman fusion on fall detection performance. Comparison holding architecture constant.}
\label{tab:kalman_vs_raw}
\centering
\begin{tabular}{llccc}
\toprule
\textbf{Input Type} & \textbf{Architecture} & \textbf{Test F1} & \textbf{Val F1} & \textbf{$\Delta$F1} \\
\midrule
Raw Gyroscope & Single-Stream & 88.96 & 92.83 & --- \\
Kalman Fusion & Single-Stream & 89.80 & 95.97 & +0.84 \\
\midrule
Raw Gyroscope & Dual-Stream & 87.58 & 92.47 & --- \\
\textbf{Kalman Fusion} & \textbf{Dual-Stream} & \textbf{91.10} & 96.10 & \textbf{+3.52} \\
\bottomrule
\end{tabular}
\end{table}

Kalman fusion improves performance in both architectures, but the benefit is dramatically larger for dual-stream (+3.52\% F1) compared to single-stream (+0.84\% F1). This synergy arises because:

\begin{enumerate}
    \item Kalman fusion transforms noisy angular velocity into stable orientation estimates
    \item Dual-stream architecture can then fully leverage these high-quality features
    \item Separate projections prevent any residual noise from corrupting acceleration features
\end{enumerate}

%-------------------------------------------------------------------
% 3.5 EMBEDDING DIMENSION ABLATION
%-------------------------------------------------------------------
\subsection{Embedding Capacity Allocation}
\label{subsec:embed_ablation}

We investigate how capacity allocation between streams affects performance. Table~\ref{tab:embed_ratio} compares different embedding splits for dual-stream architectures.

\begin{table}[H]
\caption{Effect of embedding capacity allocation in dual-stream architectures. Total embedding dimension varies with split ratio.}
\label{tab:embed_ratio}
\centering
\begin{tabular}{llcccc}
\toprule
\textbf{Input} & \textbf{Embed (acc:ori/gyro)} & \textbf{Total Dim} & \textbf{Test F1} & \textbf{Val F1} \\
\midrule
\multicolumn{5}{l}{\textit{Kalman-fused orientation:}} \\
Kalman & 32:32 (balanced) & 64 & \textbf{91.10} & 96.10 \\
Kalman & 48:48 (balanced) & 96 & 89.58 & 95.81 \\
Kalman & 48:24 (acc-heavy) & 72 & 89.05 & 96.08 \\
\midrule
\multicolumn{5}{l}{\textit{Raw gyroscope:}} \\
Raw & 48:48 (balanced) & 96 & 87.58 & 92.47 \\
Raw & 48:24 (acc-heavy) & 72 & 87.63 & 92.54 \\
Raw & 24:48 (gyro-heavy) & 72 & 88.06 & 92.33 \\
\bottomrule
\end{tabular}
\end{table}

For Kalman-fused input, balanced allocation (32:32) with total dimension 64 and channel-aware normalization (acc\_only) achieves the best test F1 (91.10\%). Increasing total capacity to 96 dimensions does not improve performance, suggesting the model is not capacity-limited but rather benefits from the regularization effect of smaller embeddings.

For raw gyroscope input, allocating more capacity to the gyroscope stream (24:48) slightly improves performance, but all raw configurations underperform Kalman-fused variants.

Figure~\ref{fig:training} shows training dynamics comparing Kalman-fused and raw input configurations.

\begin{figure}[H]
\centering
\includegraphics[width=\textwidth]{figures/training_curves_kalman_comparison.png}
\caption{\textbf{Training dynamics: Kalman-fused vs. raw input architectures.} \textbf{Top row:} Training and validation loss curves over 80 epochs. Kalman-fused models (blue/green) converge faster and achieve lower final loss compared to raw input models (orange/red). \textbf{Bottom row:} F1 score progression showing that Kalman models reach higher validation F1 earlier in training. \textbf{Key insight:} The validation-test gap is smaller for Kalman models (5.0\% vs 7.2\%), indicating better generalization. Raw gyroscope models show higher variance across folds, confirming that noisy input leads to less stable training.} {\color{blue} Validation loss should look like an L curve, the diagram did not show that}
\label{fig:training}
\end{figure}

%-----------------------------------------------------------------
%3.6 Comparison with other dual stream models
%--------------------------------------------------------------
\subsection{Comparison Across Architectures}
\label{subsec:arch_comparison}

Table~\ref{tab:arch_comparison} compares the dual-stream Kalman transformer against alternative architectures evaluated under identical LOSO-CV protocol.


\begin{table}[H]
\caption{Architecture comparison on SmartFallMM dataset. All models evaluated using 22-fold LOSO-CV. $\Delta$F1 computed relative to dual-stream Kalman transformer.}
\label{tab:arch_comparison}
\centering
\begin{tabular}{llccccc}
\toprule
\textbf{Architecture} & \textbf{Input} & \textbf{Test F1} & \textbf{Test Acc} & \textbf{Val F1} & \textbf{$\Delta$F1} \\
\midrule
\textbf{Dual-Stream Transformer} & Kalman 7ch & \textbf{91.38} & \textbf{88.44} & 95.8 & --- \\
\midrule
CNN-Mamba & Raw 6ch & 89.11 & 86.12 & 97.74 & $-$2.27 \\
CNN-Mamba & Kalman 7ch & 88.34 & 83.73 & 95.09 & $-$3.04 \\
\midrule
LSTM & Kalman 7ch & 88.84 & 85.28 & 95.54 & $-$2.54 \\
LSTM + SE + TAP & Kalman 7ch & 87.78 & 84.12 & 95.78 & $-$3.60 \\
\bottomrule
\end{tabular}
\end{table}
The dual-stream Kalman transformer outperforms alternative architectures (CNN-Mamba and LSTM) by a margin of +2.27\% to +3.60\% F1. Two observations merit discussion: (1) CNN-Mamba achieves higher performance with raw input (89.11\%) than with Kalman-fused input (88.34\%), suggesting that the Mamba state-space mechanism may perform implicit temporal smoothing that conflicts with Kalman preprocessing; (2) adding SE and TAP attention modules to the LSTM architecture decreases performance ($-$1.06\% F1), indicating that attention mechanisms designed for transformers do not transfer effectively to recurrent architectures.
%-------------------------------------------------------------------
% 3.7 COMPARISON WITH PRIOR WORK
%-------------------------------------------------------------------
\subsection{Comparison with Prior Work}
\label{subsec:sota}

Table~\ref{tab:sota} compares our approach with prior methods evaluated on the SmartFallMM dataset or similar IMU-based fall detection benchmarks.

\begin{table}[H]
\caption{Comparison with prior work on IMU-based fall detection. $^\dagger$Results on SmartFallMM dataset. $^\ddagger$Results on comparable wrist-worn IMU datasets. Protocol differences limit direct comparison.}
\label{tab:sota}
\centering
\small
\begin{tabular}{llcclc}
\toprule
\textbf{Method} & \textbf{Year} & \textbf{F1} & \textbf{Acc} & \textbf{Evaluation} & \textbf{Comparable?} \\
\midrule
\multicolumn{6}{l}{\textit{SmartFallMM Dataset:}} \\
CNN-LSTM$^\dagger$ \cite{ngu2022personalized} & 2022 & 85.2 & 88.4 & 5-fold CV & Partial \\
Personalized CNN$^\dagger$ \cite{ngu2022personalized} & 2022 & 87.8 & 90.1 & LOSO & Yes \\
\textbf{Ours$^\dagger$} & 2025 & \textbf{91.38} & 88.44 & 22-fold LOSO & --- \\
\midrule
\multicolumn{6}{l}{\textit{Other Wrist-Worn IMU Datasets (indicative only):}} \\
SVM + Handcrafted$^\ddagger$ & 2020 & 82.3 & 85.7 & LOSO & No \\
CNN-Transformer$^\ddagger$ & 2023 & 88.5 & 91.2 & 10-fold CV & No \\
\bottomrule
\end{tabular}
\end{table}

Our dual-stream Kalman transformer achieves the highest reported F1 score on SmartFallMM (91.38\% vs. prior best 87.8\%), representing a +3.58\% improvement. Our evaluation uses rigorous 22-fold LOSO-CV, while prior work may have used different subject subsets or evaluation protocols. The stricter LOSO protocol better reflects real-world generalization to unseen individuals compared to random window splits or k-fold CV that may leak subject-specific patterns.

\textbf{Accuracy vs. F1 tradeoff.} Our model achieves comparable accuracy (88.44\%) to prior methods with higher F1. This reflects our optimization objective: Focal Loss prioritizes fall detection (minority class) over overall accuracy, achieving 94.14\% recall (sensitivity) which is critical for fall detection applications where missed falls have severe consequences.

%-------------------------------------------------------------------
% 3.7 SUMMARY OF KEY FINDINGS
%-------------------------------------------------------------------
\subsection{Summary of Results}
\label{subsec:summary}

Our experiments reveal three key findings:

\begin{enumerate}
    \item \textbf{Kalman fusion is essential}: Raw gyroscope input degrades performance due to noise contamination. Kalman fusion transforms this noisy signal into stable orientation estimates, enabling +3.52\% F1 improvement in dual-stream architectures.

    \item \textbf{Dual-stream requires quality input}: Dual-stream architecture only provides benefit when input streams are high-quality. With raw gyroscope, dual-stream hurts performance ($-$1.38\% F1); with Kalman fusion, it helps (+1.30\% F1).

    \item \textbf{Channel-aware normalization matters}: Applying z-score normalization only to accelerometer channels while preserving orientation in radians improves performance by maintaining the physical meaning of orientation angles ($\phi = 0$ means upright).
\end{enumerate}

The best configuration---dual-stream Kalman transformer with SE+TAP attention, asymmetric capacity allocation (65:35), and channel-aware normalization---achieves \textbf{91.38\% test F1} $\pm$ 6.67 (Acc: 88.44\%, Prec: 89.22\%, Rec: 94.14\%, AUC: 94.30\%) on 22-fold LOSO cross-validation on SmartFallMM dataset.

\textbf{Per-subject variance analysis.} Across 22 folds, the model shows high variance ($\pm$ 6.67\% F1) due to inter-subject variability. The worst-performing subjects (F1 $<$ 85\%) share a common characteristic: they executed ``gentle'' falls with notably lower peak accelerations (mean 2.1g vs. population mean 3.4g). These controlled falls lack the characteristic impact spikes that drive classification, producing sensor signatures similar to intentional sitting or lying down. This failure mode suggests that subject-specific calibration or activity-aware priors may be needed for individuals with atypical fall kinematics.

%-------------------------------------------------------------------
% CROSS-DATASET EVALUATION
%-------------------------------------------------------------------
% Cross-Dataset Evaluation - Matching paper tone/formatting
% External validation on UP-FALL and WEDA-FALL datasets

%-------------------------------------------------------------------
% CROSS-DATASET GENERALIZATION
%-------------------------------------------------------------------
\subsection{Cross-Dataset Generalization}
\label{subsec:cross_dataset}

To evaluate generalization beyond SmartFallMM, we apply our dual-stream Kalman transformer to two external datasets with distinct sensor characteristics: UP-FALL (research-grade IMU, 18~Hz) and WEDA-FALL (consumer Fitbit, 50~Hz). Table~\ref{tab:dataset_characteristics} summarizes key differences.

\begin{table}[H]
\caption{Dataset characteristics. Sampling rate and sensor type vary significantly, requiring dataset-specific hyperparameter tuning.}
\label{tab:dataset_characteristics}
\centering
\begin{tabular}{lccc}
\toprule
\textbf{Parameter} & \textbf{SmartFallMM} & \textbf{UP-FALL} & \textbf{WEDA-FALL} \\
\midrule
Sampling Rate & 30 Hz & 18 Hz & 50 Hz \\
Sensor Type & Android Watch & Research IMU & Consumer Fitbit \\
Subjects & 51 (30Y + 21O) & 17 & 14 (young) \\
Window Size & 128 ($\sim$4.3s) & 160 ($\sim$8.9s) & 250 ($\sim$5.0s) \\
LOSO Folds & 22 & 15 & 12 \\
\bottomrule
\end{tabular}
\end{table}

%-------------------------------------------------------------------
% MAIN RESULTS TABLE
%-------------------------------------------------------------------
\subsection{Cross-Dataset Results}
\label{subsec:cross_results}

Table~\ref{tab:cross_dataset_results} presents comprehensive results across all three datasets, comparing Kalman-fused and raw input configurations.

\begin{table}[H]
\caption{Cross-dataset evaluation results. All metrics reported as mean $\pm$ std across LOSO folds. Best results per dataset in \textbf{bold}.}
\label{tab:cross_dataset_results}
\centering
\small
\begin{tabular}{llcccccc}
\toprule
\textbf{Dataset} & \textbf{Input} & \textbf{Test F1} & \textbf{Accuracy} & \textbf{Precision} & \textbf{Recall} & \textbf{AUC} & \textbf{Folds} \\
\midrule
SmartFallMM & Kalman & \textbf{91.38 $\pm$ 6.67} & \textbf{88.44} & 89.22 & \textbf{94.14} & \textbf{94.30} & 22 \\
\midrule
\multirow{2}{*}{UP-FALL}
 & Kalman & \textbf{95.14 $\pm$ 3.35} & \textbf{96.40 $\pm$ 2.57} & 94.08 $\pm$ 6.06 & \textbf{96.58 $\pm$ 3.79} & \textbf{99.14 $\pm$ 0.88} & 15 \\
 & Raw & 94.77 $\pm$ 3.29 & 96.26 $\pm$ 2.38 & \textbf{94.78 $\pm$ 5.12} & 95.06 $\pm$ 4.69 & 99.06 $\pm$ 1.05 & 15 \\
\midrule
\multirow{2}{*}{WEDA-FALL}
 & Kalman & \textbf{94.40 $\pm$ 3.21} & \textbf{92.85 $\pm$ 4.12} & \textbf{92.18 $\pm$ 4.85} & 96.89 $\pm$ 3.25 & \textbf{97.62 $\pm$ 2.84} & 12 \\
 & Raw & 90.43 $\pm$ 2.63 & 87.25 $\pm$ 3.93 & 84.91 $\pm$ 5.42 & \textbf{97.11 $\pm$ 3.20} & 94.53 $\pm$ 4.58 & 12 \\
\bottomrule
\end{tabular}
\end{table}

Kalman fusion improves test F1 on both external datasets: +0.37\% on UP-FALL and +3.97\% on WEDA-FALL. The substantially larger improvement on WEDA-FALL aligns with expectations---consumer-grade Fitbit sensors exhibit higher noise levels where Kalman filtering provides greater benefit, and the optimal 5-second window captures more complete fall dynamics.

%-------------------------------------------------------------------
% TRAIN/VAL/TEST SPLIT ANALYSIS
%-------------------------------------------------------------------
\subsection{Training Dynamics Across Datasets}
\label{subsec:train_val_test}

Table~\ref{tab:train_val_test} shows train, validation, and test metrics for each configuration, revealing generalization patterns.

\begin{table}[H]
\caption{Train/Val/Test F1 scores and loss values. $\Delta$ (Val$-$Test) indicates generalization gap.}
\label{tab:train_val_test}
\centering
\begin{tabular}{llcccccc}
\toprule
\textbf{Dataset} & \textbf{Input} & \textbf{Train F1} & \textbf{Val F1} & \textbf{Test F1} & \textbf{Val Loss} & \textbf{Test Loss} & \textbf{$\Delta$F1} \\
\midrule
\multirow{2}{*}{UP-FALL}
 & Kalman & 99.11 & 93.63 & 95.14 & 0.0004 & 0.0003 & $-$1.51 \\
 & Raw & 99.02 & 95.93 & 94.77 & 0.0002 & --- & +1.16 \\
\midrule
\multirow{2}{*}{WEDA-FALL}
 & Kalman & 98.15 & 96.42 & 94.40 & 0.0003 & 0.0003 & +2.02 \\
 & Raw & 96.41 & 94.02 & 90.43 & 0.0005 & --- & +3.59 \\
\bottomrule
\end{tabular}
\end{table}

Two observations emerge: (1) UP-FALL Kalman shows \textit{negative} $\Delta$F1, indicating test performance exceeds validation---likely due to validation subjects being harder cases; (2) WEDA-FALL Kalman achieves a smaller generalization gap ($\sim$2\%) compared to raw input ($\sim$3.6\%), indicating that Kalman fusion improves cross-subject generalization.

%-------------------------------------------------------------------
% DATASET-SPECIFIC ADAPTATIONS
%-------------------------------------------------------------------
\subsection{Dataset-Specific Hyperparameters}
\label{subsec:dataset_hyperparams}

Optimal configurations differ across datasets, reflecting sensor characteristics. Table~\ref{tab:kalman_params} shows Kalman filter noise parameters tuned per dataset.

\begin{table}[H]
\caption{Kalman filter noise parameters by dataset. Higher R values indicate noisier measurements.}
\label{tab:kalman_params}
\centering
\begin{tabular}{lcccc}
\toprule
\textbf{Dataset} & \textbf{$Q_{\text{orient}}$} & \textbf{$Q_{\text{rate}}$} & \textbf{$R_{\text{acc}}$} & \textbf{$R_{\text{gyro}}$} \\
\midrule
SmartFallMM & 0.005 & 0.01 & 0.05 & 0.10 \\
UP-FALL & 0.032 & 0.071 & 0.131 & 0.107 \\
WEDA-FALL & 0.012 & 0.132 & 0.240 & 0.282 \\
\bottomrule
\end{tabular}
\end{table}

WEDA-FALL requires the highest measurement noise parameters ($R_{\text{acc}}$, $R_{\text{gyro}}$), confirming that consumer-grade Fitbit sensors produce noisier signals than research-grade IMUs or Android smartwatches.

%-------------------------------------------------------------------
% KALMAN BENEFIT ANALYSIS
%-------------------------------------------------------------------
\subsection{Kalman Fusion Benefit by Sensor Quality}
\label{subsec:kalman_benefit}

Table~\ref{tab:kalman_benefit_sensor} directly compares the benefit of Kalman fusion across sensor quality levels.

\begin{table}[H]
\caption{Kalman fusion benefit scales inversely with sensor quality. Consumer-grade sensors benefit substantially more from noise filtering.}
\label{tab:kalman_benefit_sensor}
\centering
\begin{tabular}{llccc}
\toprule
\textbf{Dataset} & \textbf{Sensor Type} & \textbf{Kalman F1} & \textbf{Raw F1} & \textbf{$\Delta$F1} \\
\midrule
UP-FALL & Research-grade IMU & 95.11\% & 94.77\% & +0.34\% \\
WEDA-FALL & Consumer Fitbit & 94.40\% & 90.43\% & \textbf{+3.97\%} \\
\bottomrule
\end{tabular}
\end{table}

The 10$\times$ larger improvement on WEDA-FALL confirms that Kalman filtering provides greatest value when input signals are degraded by sensor noise characteristic of consumer-grade wearables.

%-------------------------------------------------------------------
% KEY FINDINGS
%-------------------------------------------------------------------
\subsection{Cross-Dataset Summary}
\label{subsec:cross_summary}

Cross-dataset evaluation validates three claims:

\begin{enumerate}
    \item \textbf{Generalization}: The dual-stream Kalman architecture achieves $>$91\% F1 across all three datasets with different sensor types and sampling rates, reaching 94.40\% on WEDA-FALL.

    \item \textbf{Kalman benefit scales with noise}: Improvement from Kalman fusion is +0.37\% on research-grade UP-FALL but +3.97\% on consumer-grade WEDA-FALL, confirming greater benefit on noisier sensors.

    \item \textbf{Hyperparameter adaptation required}: Window size and Kalman noise parameters must be tuned per dataset---UP-FALL requires 8.9s windows, while WEDA-FALL benefits from 5.0s windows to capture complete fall dynamics at 50Hz.
\end{enumerate}


%-------------------------------------------------------------------
% ARCHITECTURE ABLATION STUDY
%-------------------------------------------------------------------
\input{sections/architecture_ablation}
